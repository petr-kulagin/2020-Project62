\documentclass[12pt, twoside]{article}
\usepackage{jmlda}
\newcommand{\hdir}{.}
\usepackage[utf8]{inputenc}
\usepackage[english,russian]{babel}
\usepackage{graphicx}
\newcommand{\real}{\mathbb{R}}
\newcommand{\nat}{\mathbb{N}}
\newcommand{\integ}{\mathbb{Z}}
\usepackage{bm}
\usepackage{multicol}
%\usepackage[
%backend=biber,
%style=alphabetic,
%sorting=ynt
%]{biblatex}

%\addbibresource{Kulagin.bib}



\begin{document}

\nocite{*}

\title
    [Шаблон статьи для публикации] % краткое название; не нужно, если полное название влезает в~колонтитул
    {Построение метода динамического выравнивания многомерных временных рядов, устойчивого к локальным колебаниям сигнала.}
\author
    [И.\,О.~Автор] % список авторов (не более трех) для колонтитула; не нужен, если основной список влезает в колонтитул
    {И.\,О.~Автор, И.\,О.~Соавтор, И.\,О.~Фамилия} % основной список авторов, выводимый в оглавление
    [И.\,О.~Автор$^1$, И.\,О.~Соавтор$^2$, И.\,О.~Фамилия$^{1,2}$] % список авторов, выводимый в заголовок; не нужен, если он не отличается от основного
\email
    {author@site.ru; co-author@site.ru;  co-author@site.ru}
\thanks
    {Работа выполнена при
     %частичной
     финансовой поддержке РФФИ, проекты \No\ \No 00-00-00000 и 00-00-00001.}
\organization
    {$^1$Организация, адрес; $^2$Организация, адрес}
\abstract
    {Данная работа посвящена построению эффективного алгоритма динамического выравнивания многомерных временных рядов. Для решения данной задачи предлагается использовать функцию расстояния DTW между двумя многомерными временными рядами, согласно которому выравниваются две оси времени, при этом внутри функционала DTW выбирается расстояние между i-м и j-м измерениями такое, что оно устойчиво к локальным “сдвигам” сигнала. В качестве решения будет рассмотрено более продвинутое, основанное на DTW между парой измерений. Для проверки корректности используются как и реальные данные, например измерения активность мозга обезьян, так и искусственно сгенерированные, например движение сигнала в пространстве по часовой и против часовой стрелки.
	
\bigskip
\noindent
\textbf{Ключевые слова}: \emph {многомерные временные ряды; DTW; динамическое выравнивание.}
}

\maketitle
\linenumbers

\section{Введение}

В данной работе исследуется проблема динамического выравнивания многомерных временных рядов, устойчивого к локальным колебаниям сигнала.
Временной ряд -  собранный в разные моменты времени статистический материал о значении каких-либо параметров (в простейшем случае одного исследуемого процесса). В данном случае рассматривается многомерный случай.

Базовое решение задачи с помощью метрики L2 расстояния между рядами не всегда оказывается эффективным. Таким примером являются 2 временных ряда, полученные при близком расположении датчиков с сигналами, которые могут зафиксировать один и тот же пик. Полученный пик окажет большое влияние на значение метрики L2.

Напротив, использование известного алгоритма DTW, но уже в многомерном случае позволит обойти проблему малого расстояния между датчиками.

Полученные алгоритмы тестировались на реальных данных и искуственно  сгенерированных. Полученные результаты показали преимущество использования попарного DTW алгоритма.


\section{Постановка задачи.}

Рассматривается алгоритм решения задачи динамического выравнивания многомерных рядов.

$t = (t_1, t_2, ..., t_n), r = (r_1, r_2, ..., r_m)$, где $\forall i \in \{1, 2, ..., n\}\  X_i, Y_i \in \mathbb{R^K}$ и $\forall i  \in \{1, 2, ..., m\}  X_i \in \mathbb{R^K}$ - 2 многомерных временных ряда размерности $K$.

Для этого рассмотрим задачу двухклассовой классификации 2 многомерных временных рядов, в которой в качестве метрики используется одна из нижеприведённых.

Метрика $L_2$ между i сигналом 1 датчика и j сигналом 2 датчика: 

$d(i, j) = \Sigma_{k = 1}^K(t(i, k) - r(j, k))^2$

Расстояние DTW между парой сигналов определяется с помощью следующей рекуррентной формулы.

$D_{(i, j)} = d_{(i, j)} + min(D_{(i - 1, j)}, D_{(i, j - 1)}, D_{(i - 1, j - 1)})$

В качестве функционала ошибки будем использовать accuracy.

Будут рассматриваться искусственно сгенерированные данные, которые жестко будут разделены по положению пика на 2 класса.

Те ряды, которые имеют небольшое расстояние до пика, будут принадлежать одному классу.


\section{Базовый эксперимент}

Суть эксперимента заключается в сравнении различных метрик для сравнения многомерных временных рядов, а именно: MDTW и L2.

Эксперимент проводится в 2 этапа.

1.\textbf{Искусственно сгенерированные данные}, на которых можно увидеть достоинства и недостатки работы алгоритмов и предложить из улучшение.

2.\textbf{Реальные данные}, на которых можно протестировать полученные алгоритмы и измерить качество.

Генерация многомерных временных рядов из 1 шага проводится с помощью функций с отчётливым пиком:

$sin(x)$

$x^2$

плотность нормального распределения

В дальнейшем следует сдвиг по временной шкале семества этих функций и некоторый небольшой сдвиг максимума по шкале пространства. 

$$f(x) \rightarrow f(x + a)$$

Также будем масштабировать значения 

$$f(x) \rightarrow f(ax)$$

Суть 2 этапа заключается в проверке полученных на 1 шаге алгоритмов на реальных данных и измерением точности.

В итоге будет выбран лучший алгоритм.


 
\bibliographystyle{unsrt}
\bibliography{Kulagin}

\end{document}