\documentclass[12pt, twoside]{article}
\usepackage{jmlda}
\newcommand{\hdir}{.}

\begin{document}

\title
    [Шаблон статьи для публикации] % краткое название; не нужно, если полное название влезает в~колонтитул
    {Построение метода динамического выравнивания многомерных временных рядов, устойчивого к локальным колебаниям сигнала.}
\author
    [И.\,О.~Автор] % список авторов (не более трех) для колонтитула; не нужен, если основной список влезает в колонтитул
    {И.\,О.~Автор, И.\,О.~Соавтор, И.\,О.~Фамилия} % основной список авторов, выводимый в оглавление
    [И.\,О.~Автор$^1$, И.\,О.~Соавтор$^2$, И.\,О.~Фамилия$^{1,2}$] % список авторов, выводимый в заголовок; не нужен, если он не отличается от основного
\email
    {author@site.ru; co-author@site.ru;  co-author@site.ru}
\thanks
    {Работа выполнена при
     %частичной
     финансовой поддержке РФФИ, проекты \No\ \No 00-00-00000 и 00-00-00001.}
\organization
    {$^1$Организация, адрес; $^2$Организация, адрес}
\abstract
    {Данная работа посвящена построению эффективного алгоритма динамического выравнивания многомерных временных рядов. Для решения данной задачи предлагается использовать функцию расстояния DTW между двумя многомерными временными рядами, согласно которому выравниваются две оси времени, при этом внутри функционала DTW выбирается расстояние между i-м и j-м измерениями такое, что оно устойчиво к локальным “сдвигам” сигнала. В качестве решения будет рассмотрено более продвинутое, основанное на DTW между парой измерений. Для проверки корректности используются как и реальные данные, например измерения активность мозга обезьян, так и искусственно сгенерированные, например движение сигнала в пространстве по часовой и против часовой стрелки.
	
\bigskip
\noindent
\textbf{Ключевые слова}: \emph {многомерные временные ряды; DTW; динамическое выравнивание.}
}

\maketitle
\linenumbers

\section{Введение}
В данной работе исследуется проблема динамического выравнивания многомерных временных рядов, устойчивого к локальным колебаниям сигнала.
Временной ряд -  собранный в разные моменты времени статистический материал о значении каких-либо параметров (в простейшем случае одного исследуемого процесса). В данном случае рассматривается многомерный случай.

Базовое решение задачи с помощью метрики L2 расстояния между рядами не всегда оказывается эффективным. Таким примером являются 2 временных ряда, полученные при близком расположении датчиков с сигналами, которые могут зафиксировать один и тот же пик. Полученный пик окажет большое влияние на значение метрики L2.

Напротив, использование известного алгоритма DTW, но уже в многомерном случае позволит обойти проблему малого расстояния между датчиками.

Полученные алгоритмы тестировались на реальных данных и искуственно  сгенерированных. Полученные результаты показали преимущество использования попарного DTW алгоритма.
 
\bibliographystyle{unsrt}
\bibliography{Islamov}

\end{document}

	
\bigskip
\noindent
\textbf{Ключевые слова}: \emph {ключевое слово; ключевое слово; еще ключевые слова, разделенные <<;>>}
}

\titleEng
	[JMLDA paper template] % краткое название; не нужно, если полное название влезает в~колонтитул
    {Machine Learning and Data Analysis journal paper template}
\authorEng
	[F.\,S.~Author] % список авторов (не более трех) для колонтитула; не нужен, если основной список влезает в колонтитул
	{F.\,S.~Author, F.\,S.~Co-Author, and F.\,S.~Name} % основной список авторов, выводимый в оглавление
    [F.\,S.~Author$^1$, F.\,S.~Co-Author$^2$, and F.\,S.~Name$^{1, 2}$] % список авторов, выводимый в заголовок; не нужен, если он не отличается от основного
\thanksEng
    {The research was
     %partially
    	 supported by the Russian Foundation for Basic Research (grants 00-00-0000 and 00-00-00001).
    }
\organizationEng
    {$^1$Organization, address; $^2$Organization, address}
\abstractEng
    {This is the template of the paper submitted to the journal ``Machine Learning and Data Analysis''.
		
	\noindent
	The title should be concise and informative. Titles are often used in information-retrieval systems. Avoid abbreviations and formulae where possible.
	Please clearly indicate the last names and initials of each author and check that all names are accurately spelled. Present the authors' affiliation
	addresses where the actual work was done.
	Provide the full postal address of each affiliation, including the country name and, if available, the
	e-mail address of each author.
	Provide only institutional affiliation, department/division affiliation are not required.

	\noindent
	A concise and factual abstract is required.
	The purpose of the abstract is to provide a summary~of the paper enabling the reader to decide whether or not to read the full text.
    	The abstract should state briefly the purpose of the research, the principal results and major conclusions.
    	An abstract is often presented separately from the article, so it must be able to stand alone.
    	For this reason, References should be avoided, but if essential, then cite the author(s) and year(s).
    	Also, non-standard or uncommon abbreviations should be avoided, but if essential they must be defined at their first mention in the abstract itself.
    	The requirements on the size of the abstract is about 200--300 words.
    	It should be provided in the next structured manner:
	
	\noindent
	\textbf{Background}:	One paragraph about the problem, existent approaches and its limitations.
	
	\noindent
	\textbf{Methods}: One paragraph about proposed method and its novelty.
	
	\noindent
	\textbf{Results}: One paragraph about major properties of the proposed method and experiment results if applicable.
	
	\noindent
	\textbf{Concluding Remarks}: One paragraph about the place of the proposed method among existent approaches.
		
	\noindent
	Immediately after the abstract, provide 5-7 keywords, avoiding general and plural terms and multiple concepts (avoid, for example, ``and'', ``of'').
	Use keywords that are specific and that reflect what is essential about the paper.
	Use keywords from the abstract, introduction and conclusion.
	These keywords will be used for indexing purposes.
		
	\noindent
    	\textbf{Keywords}: \emph{keyword; keyword; more keywords, separated by ``;''}}

%данные поля заполняются редакцией журнала
\doi{10.21469/22233792}
\receivedRus{01.01.2017}
\receivedEng{January 01, 2017}

\maketitle
\linenumbers

\section{Введение}
После аннотации, но перед первым разделом,
располагается введение, включающее в себя
описание предметной области,
обоснование актуальности задачи,
краткий обзор известных результатов.

\section{Название раздела}
Данный документ демонстрирует оформление статьи,
подаваемой в электронную систему подачи статей \url{http://jmlda.org/papers} для публикации в журнале <<Машинное обучение и анализ данных>>.
Более подробные инструкции по~стилевому файлу \texttt{jmlda.sty} и~использованию издательской системы \LaTeXe\
находятся в~документе \texttt{authors-guide.pdf}.
Работу над статьёй удобно начинать с~правки \TeX-файла данного документа.

Обращаем внимание, что данный документ должен быть сохранен в кодировке~\verb'UTF-8 without BOM'.
Для смены кодировки рекомендуется пользоваться текстовыми редакторами \verb'Sublime Text' или \verb'Notepad++'.

\paragraph{Название параграфа}
Разделы и~параграфы, за исключением списков литературы, нумеруются.

\section{Заключение}
Желательно, чтобы этот раздел был, причём он не~должен дословно повторять аннотацию.
Обычно здесь отмечают, каких результатов удалось добиться, какие проблемы остались открытыми.

%%%% если имеется doi цитируемого источника, необходимо его указать, см. пример в \bibitem{article}
%%%% DOI публикации, зарегистрированной в системе Crossref, можно получить по адресу http://www.crossref.org/guestquery/
\begin{thebibliography}{99}
\bibitem{book}
    \BibAuthor{Гуссенс~М., Миттельбах~Ф., Cамарин~А.}
    \BibTitle{Путеводитель по пакету \LaTeX\ и~его расширению \LaTeXe} / Пер. с англ.~---
    М.:~Мир, 1999. 606~с.
    (\BibAuthor{Goossens M., Mittelbach F., Samarin A.}
     \BibTitle{The \LaTeX\ companion}.~--- 2nd ed.~--- Reading, MA, USA: Addison-Wesley, 1994. 528 p.)

\bibitem{article}
    \BibAuthor{Загуренко~А.\,Г., Коротовских~В.\,А., Колесников~А.\,А., Тимонов~А.\,В., Кардымов~Д.\,В.}
    Технико-экономическая оптимизация дизайна гидроразрыва пласта~//
    \BibJournal{Нефтяное хозяйство}, 2008. Т.~11. \No\,1. С.~54--57.
	\BibDoi{10.3114/S187007708007}.

\bibitem{webArticle}
	\BibAuthor{Blaga~P.\,A.}
	Commutative Diagrams with XY-pic II. Frames and Matrices~//
	\BibJournal{PracTEX J.}, 2007. Vol.\,4.
	URL: \BibUrl{https://tug.org/pracjourn/2007-1/blaga/blaga.pdf}.

\bibitem{webResource}
	XYpic.
	URL: \BibUrl{http://akagi.ms.u-tokyo.ac.jp/input9.pdf}.
	
\bibitem{inproceedingsRus}
	\BibAuthor{Усманов~Т.\,С., Гусманов~А.\,А., Муллагалин~И.\,З., Мухаметшина~Р.\,Ю., Червякова~А.\,Н., Свешников~А.\,В.}
	Особенности проектирования разработки месторождений с применением гидроразрыва пласта~//
	\BibJournal{Труды 6-го Междунар. симп. <<Новые ресурсосберегающие технологии недропользования и повышения нефтегазоотдачи>>}.~---
	М.:~Издательство, 2007. С.~267--272.

\bibitem{inproceedingsEng}
    \BibAuthor{Author~N.}
    Paper title~//
    \BibJournal{10th Conference (International) on Any Science Proceedings}.~---
    Place of publication: Publisher, 2009. P.~111--122.

\bibitem{techreport}
	\BibAuthor{Lambert~P.}
  	\BibTitle{The title of the work}.
  	Place of publication:~The institution that published, 1993.  Report~2.
 	
\end{thebibliography}

%%%% если имеется doi цитируемого источника, необходимо его указать, см. пример в \bibitem{article}
%%%% DOI публикации, зарегистрированной в системе Crossref, можно получить по адресу http://www.crossref.org/guestquery/.

\maketitleSecondary
\English
\begin{thebibliography}{99}
\bibitem{book}
	\BibAuthor{Goossens,~M., F. Mittelbach, and A.~Samarin}. 1994.
	\BibTitle{The \LaTeX\ companion}.
	2nd ed.
	Reading, MA: Addison-Wesley. 528 p.

\bibitem{article}
	\BibAuthor{Zagurenko,~A.\,G., V.\,A.~Korotovskikh, A.\,A.~Kolesnikov, A.\,V.~Timonov, and D.\,V.~Kardymon}. 2008.
	Tekhniko-ekonomicheskaya optimizatsiya dizayna gidrorazryva plasta
	[Technical and economic optimization of the design of hydraulic fracturing].
	\BibJournal{Neftyanoe Khozyaystvo} [Oil Industry] 11(1):54--57.
	\BibDoi{10.3114/S187007708007}. (In Russian)

\bibitem{webArticle}
	\BibAuthor{Blaga,~P.\,A.} 2007.
	Commutative Diagrams with XY-pic II. Frames and Matrices.
	\BibJournal{PracTEX J.}  4.
	Available at: \BibUrl{https://tug.org/pracjourn/2007-1/blaga/blaga.pdf}
    (accessed February 20, 2007).

\bibitem{webResource}
	XYpic.
	Available at: \BibUrl{http://akagi.ms.u-tokyo.ac.jp/input9.pdf}
	(accessed April 09, 2015).

\bibitem{inproceedingsRus}
	\BibAuthor{Usmanov,~T.\,S., A.\,A.~Gusmanov, I.\,Z.~Mullagalin, R.\,Yu.~Mukhametshina, A.\,N.~Chervyakova, and A.\,V.~Sveshnikov.} 2007.
	Osobennosti proektirovaniya razrabotki mestorozhdeniy s primeneniem gidrorazryva plasta
	[Features of the design of field development with the use of hydraulic fracturing].
	\BibJournal{6th Symposium (International) ``New Energy Saving Subsoil Technologies and the
	Increasing of the Oil and Gas Impact'' Proceedings}.
	Moscow:~Publisher. 267--272. (In Russian)
	   	
\bibitem{inproceedingsEng}
    \BibAuthor{Author,~N.} 2009.
    Paper title.
    \BibJournal{10th Conference (International) on Any Science Proceedings}.
    Place of publication: Publisher. 111--122.
	
\bibitem{techreport}
	\BibAuthor{Lambert,~P.} 1993.
  	\BibTitle{The title of the work}.
  	Place of publication:~The institution that published.  Report~2.
  	     	
\end{thebibliography}

\end{document}
